\documentclass[12pt, a4paper, onecolumn, oneside]{report}
\usepackage[bahasa]{babel}
\usepackage[utf8]{inputenc}
\usepackage{geometry}
\usepackage{setspace}
\usepackage{titlesec}
\usepackage{graphicx}
\usepackage{tabularx}
\usepackage{booktabs}
\usepackage{hyperref}
\usepackage{indentfirst}
\usepackage{tocloft}
\usepackage{titletoc} % Add this package for more TOC customization
\usepackage{mathptmx} % Times New Roman font

% Margin settings
\geometry{
    left=3cm,
    right=2.5cm,
    top=3cm,
    bottom=2.5cm
}

% Chapter numbering in Roman numerals
\renewcommand{\thechapter}{\Roman{chapter}}

% Section and subsection numbering in Arabic numerals
\renewcommand{\thesection}{\arabic{chapter}.\arabic{section}}
\renewcommand{\thesubsection}{\arabic{chapter}.\arabic{section}.\arabic{subsection}}

% Chapter title format
\titleformat{\chapter}[display]
{\normalfont\bfseries\centering\fontsize{14}{16.8}\selectfont}
{\chaptertitlename\ \thechapter}{14pt}{\fontsize{14}{16.8}\selectfont}

% Adjust chapter spacing to remove extra vertical space
\titlespacing*{\chapter}{0pt}{0pt}{20pt}

% Section title format
\titleformat{\section}
{\normalfont\bfseries\fontsize{12}{14.4}\selectfont}
{\thesection}{1em}{}

% Subsection title format
\titleformat{\subsection}
{\normalfont\bfseries\fontsize{12}{14.4}\selectfont}
{\thesubsection}{1em}{}

% Line spacing 1.15
\setstretch{1.15}

% Table of Contents, List of Figures, and List of Tables formatting
\usepackage{etoolbox}
\renewcommand{\cfttoctitlefont}{\hfil\Large\bfseries}
\renewcommand{\cftaftertoctitle}{\hfil\par}
\renewcommand{\cftloftitlefont}{\hfil\Large\bfseries}
\renewcommand{\cftafterloftitle}{\hfil\par}
\renewcommand{\cftlottitlefont}{\hfil\Large\bfseries}
\renewcommand{\cftafterlottitle}{\hfil\par}

% Dotted leaders settings
\makeatletter
\renewcommand{\@dotsep}{2}
\makeatother

% Customize TOC entries with dotted leaders
\titlecontents{chapter}[0pt]
{\addvspace{0pt}\bfseries}
{\contentslabel{0pt}}
{\hspace*{0pt}}
{\titlerule*[0.5pc]{.}\contentspage}

\titlecontents{section}[1em]
{\addvspace{0pt}}
{\contentslabel{1em}}
{\hspace*{0pt}}
{\titlerule*[0.5pc]{.}\contentspage}

\titlecontents{subsection}[2em]
{\addvspace{0pt}}
{\contentslabel{2em}}
{\hspace*{0pt}}
{\titlerule*[0.5pc]{.}\contentspage}

\begin{document}
\chapter{Asta AI Markdown Editor}

Welcome to Asta AI Markdown Editor, where coding meets cuteness! ✨ This isn’t just another boring Markdown editor—it’s your new best friend for organizing projects, writing docs, and sprinkling a little AI magic into your workflow. Built with love on \textbf{FastAPI}, \textbf{SQLite}, and pure \textbf{Vanilla JS/CSS/HTML}, Asta is here to make your life easier—and way more fun.

---

\subsection{🎉 What’s Inside the Magic Box?}

\begin{figure}[ht]
  \centering
  \includegraphics[width=0.8\textwidth]{/files/test/pasted_image_1745761030756.jpg}
  \caption{pasted_image_1745761030756.jpg}
  \label{fig:pasted_image_1745761030756}
\end{figure}


\textbf{Split View Magic}
Write your raw Markdown on the left, and watch it come to life on the right! It’s like having a tiny wizard in your browser, casting spells to turn your words into beautiful docs.

\textbf{AI Brainstorm Buddy}
Stuck? Let the AI take over! The bottom AI pane will continue your thoughts without messing up your work. It’s like having a coding buddy who’s always ready to help—but without the coffee breath.

\textbf{Rewrite Wizardry}
Highlight some text, right-click, and boom! The AI will polish your words like a pro. Perfect for when you want to sound smart without actually trying.

\textbf{Notes for Nerds}
Got extra context or random thoughts? Throw them into the Notes panel, and the AI will use them to make your writing even better. It’s like giving your AI a cheat sheet!

\textbf{Workspace Wonderland}
Create new projects with a single click. No project? No problem! You can still work, but saving is disabled—so don’t forget to create one before you get too deep into your genius ideas.

\textbf{Image Paste Party}
Copy-paste images directly into the editor, and they’ll be saved in your project directory. It’s like magic, but with fewer rabbits and more Markdown.

\textbf{Save & Autosave}
Your work is safe with Asta. Save manually or let autosave do its thing—either way, your genius won’t be lost.

---

\subsection{🛠️ How to Get Started}

\begin{enumerate}
\item \textbf{Clone the Repo}
   \begin{verbatim}
git clone \url{https://github.com/Iteranya/Asta-AI-Editor.git}
   cd Asta-AI-Editor
\end{verbatim}
\end{enumerate}

\begin{enumerate}
\item \textbf{Install the Goodies}
   \begin{verbatim}
pip install -r requirements.txt
\end{verbatim}
\end{enumerate}

\begin{enumerate}
\item \textbf{Run the Server}
   \begin{verbatim}
uvicorn main:app --reload
\end{verbatim}
\end{enumerate}

\begin{enumerate}
\item \textbf{Open Your Browser}
\end{enumerate}
   Head to \texttt{\url{http://localhost:5452}} and let the fun begin!

---

\subsection{🧙‍♂️ Tech Stack}

\begin{itemize}
\item \textbf{Backend}: FastAPI, SQLite, OpenAI API
\item \textbf{Frontend}: Vanilla JavaScript, HTML, CSS
\end{itemize}

---

\subsection{📄 License}

GPL-3 License. Use it, tweak it, build on it—just don’t gatekeep it!

\end{document}